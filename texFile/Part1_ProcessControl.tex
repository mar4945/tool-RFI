Lane detection remains a cornerstone within the realm of advanced driver assistance systems, representing a critical element for intelligent autonomous systems and applications in smart vehicles. Notably, Lane Departure Warning Systems (LDWS) emerge as indispensable safety mechanisms, serving to alert drivers when their vehicles deviate from designated lanes on highways [1]. Over recent years, lane detection methodologies have undergone extensive scrutiny, leading to a classification of current studies into three primary categories: feature-based methods [2], [3], [4], model-based approaches [5], [6], and deep learning techniques [7], [8].

Feature-based methodologies typically entail the recognition of lanes through the analysis of lane marking attributes such as color and line edges. Employing techniques such as contrast enhancement and edge detection, these methods aim to detect lane markings through thresholding methods or Hough Transform (HT) variants [9], [10]. However, a notable drawback of feature-based methods lies in their dependency on clearly delineated lane marks, rendering them susceptible to disruptions such as weak lane markings and occlusions. To address these limitations, the concept of Inverse Projection Mapping (IPM) has been introduced [11]. By generating a bird's-eye view of the road surface, IPM circumvents issues associated with obscured or faint lane markings, although its efficacy relies heavily on the assumption of a perfectly flat road surface and precise camera calibration.

In model-based methods, lane markings are identified through the modeling of lanes using predetermined models such as straight-line, parabolic, or spline models. Various algorithms have been proposed, including combinations of Dynamic Programming (DP) and Hough Transform [5], as well as curve model fitting methods in conjunction with gradient enhancement and edge detection techniques [12]. However, the complexity of developing reliable models poses a significant challenge for model-based approaches.

Deep learning techniques, on the other hand, leverage deep learning algorithms, particularly Artificial Neural Networks (ANNs), to extract lane information. These methods, exemplified by the integration of Convolutional Neural Networks (CNNs) with algorithms like RANdom SAmple Consensus (RANSAC) [7] or Recurrent Neural Networks (RNNs) [8], demonstrate promising capabilities in noise reduction and robust lane detection, especially in scenarios with challenging conditions such as bad lane markings or vehicle occlusion.

Despite their efficacy, the aforementioned state-of-the-art methods often entail high computational demands, particularly deep learning approaches. In response, this paper proposes a novel Lane Detection algorithm based on Iterative Tree Search (ITS), designed specifically for low-cost hardware deployment. Characterized by its speed, computational efficiency, and low power consumption, the ITS-based approach represents a significant contribution to the field, offering a promising alternative to existing methodologies. Notably, to the best of our knowledge, there are no other algorithms for lane detection based on Iterative Tree Search, further highlighting the novelty and potential impact of this research endeavor.


STOOOOOOOOOOOOOOOOOOOOP

Path following constitutes a significant application challenge within the domain of Unmanned Aerial Vehicles (UAVs), holding particular relevance in precision agriculture settings where robust path following algorithms are essential for maintaining high productivity rates and facilitating optimal plant growth [1]. Moreover, in civilian applications such as power line monitoring, path following encapsulates the essential task of navigating between multiple target regions requiring inspection [2]. Irrespective of the specific application domain, the successful completion of missions by drones hinges upon their ability to safely and accurately follow predefined paths.

Examining the path following problem reveals two distinct components: path detection and path following [3], [4]. In terms of path detection, the Hough transform and its advancements are widely regarded as prominent solutions in the literature [5], [6], [7]. However, the computational demands associated with such transformations pose challenges, particularly for on-board implementation in aircraft with stringent battery and processing constraints. These challenges are further exacerbated in the context of Micro Aerial Vehicles (MAVs) where both sensor equipment and vehicle dimensions are minimal. Conversely, lightweight machine learning solutions have emerged as promising alternatives for path detection, albeit the time required for algorithm setup impedes their practical application in real-world scenarios [8], [9]. Hence, there is a growing interest in low computational intensity algorithms capable of providing path following references within predefined time constraints, even without prior knowledge of the surrounding environment.

Transitioning to the path following stage, contemporary solutions predominantly leverage nonlinear guidance laws, vector fields, and pure pursuit algorithms owing to their simplicity and ease of implementation [10], [11]. While the choice of path planner is contingent upon specific application requirements, certain general considerations can be made. Nonlinear guidance laws, while simple to implement, exhibit degraded performance in scenarios characterized by rapid changes in target acceleration, leading to significant trajectory generation delays [12]. Consequently, ensuring stability necessitates a comprehensive understanding of target velocity and acceleration dynamics. Conversely, while vector field solutions mitigate oscillation issues inherent in nonlinear guidance laws, they entail substantial computational overhead [13]. Meanwhile, the pure pursuit approach offers a viable solution for scenarios where tracking accuracy and computational efficiency are paramount. By dynamically adjusting the position of a look-ahead point based on predefined tracking criteria, pure pursuit algorithms aim to minimize the distance between the current position and the anticipated path, thereby facilitating precise path following [15]–[17].

STOOOOOOOOOOOOOOOOOOOOOOOOOOOOOOOOOOOOOOOOOOOOOOOP

Autonomous driving stands as one of the most promising prospects in today's automotive landscape [1]. Nevertheless, vehicles often navigate through intricate scenarios, where various actors with highly unpredictable behaviors coexist. To navigate such complex environments effectively, autonomous vehicles must process multiple facets of their surroundings, forecast environmental evolution, and devise suitable strategies or policies based on predefined objectives. Analogous to human decision-making processes, the predictive abilities and policy selection of autonomous vehicles should draw from their past learning experiences.

Reinforcement Learning (RL), a subfield of Machine Learning, presents a broad spectrum of computational approaches for goal-directed learning through interaction [2]. Particularly in recent years, when combined with Deep Learning, RL has found successful applications in numerous typical driving scenarios involving actors with unpredictable behaviors, such as other vehicles or pedestrians [3], [4]. However, safety considerations impose constraints on these applications, necessitating a careful balance between the inherent benefits of experience-driven learning methods and adherence to safety protocols [5], [6]. Notably, it becomes increasingly prudent to initially train RL algorithms offline via simulators before transitioning to online learning in the actual operational environment.

The unexpected crossing of pedestrians represents one of the most common and critical scenarios in urban areas. According to data from the National Center for Statistics and Analysis [7], approximately 6,000 pedestrians were fatally injured along roadways in the United States in 2017. Given the advancement of autonomous vehicles, the implementation of pedestrian collision avoidance systems assumes paramount importance. This work proposes an approach grounded in Deep Reinforcement Learning for autonomous vehicles navigating scenarios marked by unforeseen pedestrian crossing events. In addition to the imperative of pedestrian avoidance, the resulting agent must adhere to a predefined trajectory.

We adopt a continuous-time state/action space representation for the problem at hand, motivating the utilization of the Deep Deterministic Policy Gradient (DDPG) algorithm for agent training [8]. Both training and testing of the DDPG-based agent are conducted via numerical simulations (i.e., offline) to ensure compliance with aforementioned safety constraints. The proposed approach offers several advantages inherent to RL-based methods, including the ability to handle complex systems or scenarios difficult to model, learning via simulators or real-world interaction, adaptability of learned policies to environmental changes, and minimal time required for control action selection from the learned policy [2], [9].

While various RL-based approaches have been applied to autonomous driving systems in literature, differences exist compared to our approach, such as the utilization of discrete action spaces that may not accurately represent real car control inputs, differences in problem formulation, and the adoption of online RL approaches [10], [11].


STOOOOOOOOOOOOOOOOOOOP

Autonomous Vehicle (AV) technologies represent a glimpse into the future of transportation, with many car manufacturers integrating new functionalities that constitute the foundation of Advanced Driver Assistance Systems (ADAS), ultimately paving the way for full autonomous driving capabilities [1]. However, amidst this progress, there are significant concerns regarding cybersecurity, which stands to become increasingly pivotal as Internet connectivity emerges as a critical enabling technology [1]. The susceptibility of Control Area Network (CAN) based communication and internet connections exposes vehicles to potential malicious attacks targeting onboard control units.

While the computer science community has extensively addressed cybersecurity in AVs, there remains a notable absence of contributions from the control systems perspective [2]. In efforts to address this gap, recent research endeavors have implemented Extended Kalman Filter (EKF) based architectures to mitigate potential sensor attacks [3], [4]. By employing a residual cumulative sum detector to monitor deviations between predicted and measured states, these approaches offer a means of detecting cyber-attacks. However, the reliance on a single threshold for detection raises concerns regarding false alarms, and strategies for mitigating the impact of detected attacks remain unexplored.

Expanding the purview of cybersecurity analysis from a control systems perspective, significant attention has been directed towards scenarios wherein the plant under scrutiny is modeled as a Linear-Time-Invariant (LTI) system with corresponding control loops. Notably, in [5], the estimation and control of linear systems under cyber-attacks are examined, with the authors proposing an algorithm based on compressed sensing for reconstructing corrupted states. However, limitations emerge, as demonstrated by the impossibility of reconstructing the state of an LTI system if more than half of the sensors are compromised.

Model Predictive Control (MPC) based solutions have also been explored to bolster the resilience of control loops in the face of cyber threats, as evidenced in [6] and [7]. These studies implement LTI-MPC controllers to enhance control loop robustness when the threatened plant is assumed to be linear and time-invariant. However, while assuming LTI dynamics may not always align with the complexities of AVs, it offers mathematical tractability, which proves advantageous for investigating the problem at hand.