\chapter{Introduction}
\newcommand{\LR}[1]{\textcolor{magenta}{#1}}
\lhead{\bfseries INTRODUCTION}

\label{cap:introduction}
The first two Sections of this Chapter aim to provide the main purposes of this work, and the outline of this Thesis, respectively. Some of the topics in the present dissertation were presented both in national and international conferences and papers. The last Section of this Chapter contains the list of the Author's publications derived from the present work. 
\section{Background and motivations}
In everyday life, we are surrounded by mechanisms and processes that are controlled or monitored to achieve automated, efficient and safe operations. For example, the temperature in our homes is controlled by a thermostat, the speed of our cars by a cruise control system, and a wide range of manufacturing equipment is monitored to ensure that it is operating safely and efficiently.\\
\textbf{Process control} consists of continuously acting on the process, selecting one or more inputs in such a way as to cause certain process outputs to behave in a desired manner, whether to remain close to a set reference value or to \textit{track} a desired \textit{trajectory}, usually optimizing a given performance index. Process control systems typically consist of three main components arranged in a closed loop configuration (Fig.\ref{fig:Control System}):
\begin{itemize}
    \item A sensor that measures the output of the process
    \item A controller that compares the measured output to the desired output and generates a control signal
    \item An actuator that implements the control signal by adjusting one or more of the inputs to the process,
  
\end{itemize}

\begin{figure}[ht!]
    \centering
    \includegraphics[scale=0.75]{figure/Introduction/Process_control.png}
    \caption{Process Control System Scheme}
    \label{fig:Control System}
\end{figure}
%\LR{At the very essence of many process control systems there is the  \textbf{feedback} concept \textit{i.e.} information about the output of a system is used to adjust its input, creating a loop that helps regulate and improve the system's behavior.  It continuously corrects deviations between actual and desired outputs, preventing instability and enhancing system performance and resilience to uncertainties, disturbances, and parameter variations}.

At the core of  process control systems is the \textbf{feedback} concept, where information about the system's output is used to adjust its input, creating a loop that continuously corrects deviations, enhances performance, and improves resilience. 
\textbf{Process monitoring }is the continuous surveillance of a process for signs of malfunction. The goal of process monitoring is to detect problems (\textit{faults} or \textit{anomalies}) at an early stage so that corrective action can be taken before the problem causes damage or downtime.
Process monitoring systems include at least one sensor that has been conditioned and pre-processed to emphasize certain essential features and reduce noise levels. The pre-processed data is then analyzed by a computer to identify patterns that could indicate a problem.
In this case, the control action is discrete and the selection based on the identified features is usually reduced to a simple classification task Fig.\ref{fig: Process Monitoring System}. The objective, or \textit{track}, in this case, is to ensure the safe and correct operation of the plant.
\begin{figure}[ht]
    \centering
    \includegraphics[scale=0.75]{figure/Introduction/Process_monitoring.png}
    \caption{Process Monitoring system scheme}
    \label{fig: Process Monitoring System}
\end{figure} \\
\textbf{Model-based control theory} is the milestone of many  traditional control and monitoring systems , allowing the design of stable and predictable systems that provide guarantees for known operating conditions, by exploiting a given dynamical model of the process. However, the increasing complexity of applications makes it difficult to have a precise process model, so decisions often have to be made with only partial knowledge of the world.  

\textbf{Data-driven learning techniques} can be used to adapt to changes in the system.
In the last decade, a large body of work has explored the different ways these methods can be applied in both areas, extending the boundaries for system optimization and performance enhancement. By leveraging data-driven techniques, researchers and practitioners have unlocked new possibilities for fine-tuning control strategies and improving the efficiency of monitoring systems. These approaches harness the power of real-time data collection, analysis, and interpretation to adjust parameters dynamically, predict system behavior, and respond to variations that might otherwise go unnoticed.
The contributions fall into two broad categories that can be referred as:
\begin{enumerate}
    \item \textit{Learning as  an aid}, where the integration of learning in model-based techniques allows the exploitation of the guarantees that come with model-based approaches and extends the limits imposed by these approaches, allowing some adaptation based on learning techniques.
    \item \textit{Learning as an alternative} where learning techniques are presented as an alternative approach to control theory for solving monitoring and control problems, guaranteeing more flexibility but making it hard to guarantee safety.
\end{enumerate}    
Central to this research is the following question:\begin{center}
\textit{ What are the shortfalls of model-based control theory and how learning-based techniques can be applied to improve control and monitoring systems?}  
\end{center}

\section{Addressed Problems}
This general question is addressed with the aim of developing innovative algorithms and solutions for problems that arise in robotics and real industrial plant monitoring. In the following, we give a general definition of the  addressed problems, indicating also the assumption we have made for approaching the solution and the application-related difficulties.   
%\LR{Introduce here, in particular, the problem that has been addressed in the context of this general question, particularly for robotic applications such as path planning and path following, and the problem of  industrial process monitoring, trying to give a formal definition of these problems and then put here the part related to  Dissertation outline and contribution and the list of publication }
\subsection{Path Following Problem for UAVs}
In recent years, the interest in developing fully autonomous aerial vehicles, or Unmanned Aerial Vehicles (UAVs), has shifted towards civil applications, surpassing military demand. Key innovative applications include UAV interaction with the environment, such as infrastructure maintenance and parcel delivery, with a focus on control and disturbance rejection. While supervision and mapping remain primary applications, UAVs are increasingly utilized in environmental protection, security, agriculture, and infrastructure supervision. Multirotors, known for maneuverability and stability, are prominent among UAVs. Initially, efforts for multirotors centered on stabilizing attitude dynamics, utilizing various techniques. With stabilization control well-established, the current challenges involve trajectory control, fault-tolerant control, path planning, and obstacle avoidance. Trajectory control aims to make the vehicle follow a predefined path, and can  be solved mainly by two different approaches : trajectory tracking and path following controllers.  For the trajectory tracking problem a reference specified in time is tracked, where the references of the path are given by a temporal evolution of each space coordinate. Path following controllers, which do not require preassigned timing information, offer advantages such as easier design, smoother convergence to the path, reduced control effort, smaller transient error, enhanced robustness, and a lower likelihood of control signal saturation. \cite{} \cite{}.
Forth 2-D case , the path-following problem for an  Unmanned Aerial Vehicle UAV can be formulated as follows.
Given the initial position of the UAV on a plane $p=(x,y)$ and its heading angle $\psi$ we want to determine the commanded heading angle for the vehicle such that the vehicle accurately tracks the path. The general assumption is that the vehicle starts with a lateral distance d from the path, which can be expressed as a point line distance. If the path is a straight line (like the case under analysis in our paper) then the Objective of the path following problem can be expressed as the following condition, say that $\theta$ is some of the angle of the straight line that passes between two waypoints on the path than we want on the long period that $|\psi-\theta|\rightarrow0$ and $|d|\rightarrow0$.  In solving this problem  we rely on two assumptions:
\begin{itemize}
    \item We start along the path.
    \item The only information that we have about the path come from the down facing camera.
\end{itemize}
For a forward looking camera the computer vision task can be more 
\subsection{Path planning Problem}
The path planner is a high-level controller whose mission is generally to obtain an optimal path between two given points in space. The resulting optimal path, which commonly seeks a
compromise between a minimum path length and a minimum
control effort will vary based on the problem requirements.
A path planning problem is completely specified by the following elements:
\begin{itemize}
    \item  A \textbf{World model} $\mathcal{W}$  describes the robot workspace and its boundary determines the obstacles $\mathcal{O}_i$\footnote{$
2 D \text { world, } \mathcal{w}=\mathbb{R}^2
$}
\item  A \textbf{Robot} is defined by its geometry, and parameters (kinematics) and it is controllable by the motion plan.
\item   A \textbf{Configuration space}  $\mathcal{C}$  ( $\mathcal{C}$-space)  to describe possible configurations of the robot. The robot's configuration completely specifies the robot location in $\mathcal{W}$ including the specification of all degrees of freedom \footnote{$
\text { E.g., for a material point } \mathcal{C}=\{x, y\} \text{ while for  a robot with rigid body in a plane } \mathcal{C}=\{x, y, \varphi\}=\mathbb{R}^2 \times S^1 \text {. }$}. Specifically we can define:
\begin{itemize}
    \item $\mathcal{A}$ a subset of $\mathcal{W}$ occupied by the robot, $\mathcal{A}=\mathcal{A}(q)$.
    \item $\mathcal{C}_{\text {obs }}$  subset of $\mathcal{C}$ occupied by obstacles given by $
    \mathcal{C}_{\text {obs }}=\left\{q \in \mathcal{C}: \mathcal{A}(q) \cap \mathcal{O}_i, \forall i\right\} .
    $
    \item Collision-free configurations are  $
    \mathcal{C}_{\text {free }}=\mathcal{C} \backslash \mathcal{C}_{\text {obs }}
    $
\end{itemize}
\item A \textbf{Path} is a continuous mapping in $\mathcal{C}$-space such that $\pi:[0,1] \rightarrow \mathcal{C}_{\text {free }}$, with $\pi(0)=q_0$, and $\pi(1)=q_f$
\end{itemize}

\subsection{Fault detection for industrial processes}
A Fault Detection (FD) system relies on process measurements, typically captured by sensor signals denoted as vector $y$. If control signals $(u$ ) regulate the process operation, they can also serve as measurements for FD. We represent sensor signals as $y$ and control signals as $u$. Let $f$ be a signal vector representing the fault, following the conditions: $f=0$ for fault-free and $f \neq$ for faulty situations.
The initial step in effective fault detection involves creating a mapping from the measurement space $(y, u)$ to the fault's image space $(f)$, applied as a fault detector. This process is described as $J=\mathcal{J}(y, u)=\mathcal{D}(f)$.
The term for the function $J$ varies; in machine learning, it is called a feature, while in statistical multivariate analysis (MVA) and model-based fault diagnosis, it is referred to as a test statistic and evaluation function, respectively.
In the model-based FD framework, especially when the process is driven by control input $u$, $\mathcal{D}(f)$ is often constructed in two steps: (1) mapping $(y, u)$ to the residual subspace with the residual vector $r$ (also an image space of $f$, and (ii) building $J$ as a function of $r$. Mathematically these steps are expressed as (i) $r=\mathcal{K}(y, u)=\mathcal{Q}(f)$ and (ii) $J=\mathcal{J}(r)=\mathcal{J}(\mathcal{Q}(f))=$ $\mathcal{D}(f)$. The primary challenge in fault diagnosis arises from process uncertainties, expressed as additive unknown inputs, including noises and deterministic disturbances. To distinguish between the influences of unknown inputs and the fault $f$, two key steps are widely adopted: (1) designing the function $J$ to amplify the influence of $f$ on the fault detector and simultaneously diminish the influence of unknown inputs, and (i) introducing a threshold $J_{t h}$. Based on $\left\{J, J_{t h}\right\}$, a simple detection logic is implemented online for fault detection: $J-J_{t h} \leq 0$ implies fault-free, while $J-J_{t h}>0$ implies faulty, triggering an alarm. Upon detecting a fault, a fault estimator can be activated as needed. It is driven by either $y$ or $(y, u)$ and can be described as $\hat{f}=\mathcal{I}(y)$ or $\hat{f}=\mathcal{I}(y, u)$. In summary, the main design tasks for a fault detector and a fault estimator involve constructing $\mathcal{J}(y, u)$ or $\mathcal{K}(y, u)$ and $\mathcal{J}(r)$, setting $J_{t h}$, and constructing $\mathcal{I}(y)$ or $\mathcal{I}(y, u)$. 
\section{Dissertation outline }
\textbf{Chapter \textcolor{red}{\ref{Chapter: 1}}: Literature review.} The literature review chapter serves as a foundation for the research by exploring the theoretical and conceptual underpinnings of robotics, control, and learning-based control. It systematically reviews the existing body of knowledge, highlighting key concepts and methodologies. In addition, the chapter critically analyses related work, identifying gaps and establishing the unique positioning of the research within the broader academic landscape. By reviewing previous research, the chapter aims to justify the need for the proposed contributions and lay the groundwork for the subsequent detailed discussions in the remaining chapters.

\textbf{Part \textcolor{red}{\ref{part:1}} Process Control and Learning}
\begin{itemize}
    \item \textbf{Chapter \textcolor{red}{\ref{Chapter:2}}:Embedded Perception and
control for basic path following }
This chapter focuses on the specific application of vision-based path following for UAVs with a down-facing camera. It begins with a clear definition of the problem and the limitations of existing approaches. The proposed solution, based on classical PI controller and an image-based visual servoing algorithm, is presented in detail. Results from numerical simulations and validation using MATLAB and The MathWorks Virtual Reality (VR) toolbox underscore the validity and effectiveness of the proposed solution. On the perception side, we introduce also a novel lane detection method for a front-facing camera based on iterative-search trees that can serve for further extension of the proposed path-following algorithm for real world application.
\item \textbf{Chapter \textcolor{red}{\ref{Chapter:3}}: Reinforcement learning for trajectory tracking and pedestrian collision-avoidance}: In this Chapter we introduce a pure Learning-based approach for autonomous driving applications. It addresses the challenges of pedestrian collision avoidance and trajectory tracking using the Reinforcement Learning framework. The application of the Deep Deterministic Policy Gradient algorithm is discussed, emphasizing its role in endowing the autonomous agent with the capabilities to manage unexpected scenarios and track specific trajectories. Numerical simulations further validate the effectiveness of the proposed Reinforcement Learning system.
\item \textbf{Chapter \textcolor{red}{\ref{Chapter:4}}: Embedded Path planning using fast learning based MPC } This Chapter introduces a supervised learning framework for solving multi-parametric Mixed Integer Linear Programs (MILPs) arising in Model Predictive Control. The innovative approach, inspired by Branch-and-Bound techniques, involves training a Neural Network/Random Forest to predict strategies for solving MILPs. The chapter concludes with a demonstration of the approach in a motion planning example, comparing the results against various commercial and open-source mixed-integer programming solvers.

\end{itemize}



\textbf{Part \textcolor{red}{\ref{part:2}} Process Monitoring and Learning}
\begin{itemize}
    \item \textbf{Chapter \textcolor{red}{\ref{Chapter:5}}}: This Chapter delves into the application of an Unsupervised learning framework, for fault detection in the steel industry. The chapter details the  one-class support vector machines (OC-SVM) approach, its implementation, and the validation using production data from a steel-making industry in Southern Italy. Comparative analysis with a multivariate statistical method highlights the superior performance of OC-SVM, particularly in predicting breakdowns.
    \item \textbf{Chapter \textcolor{red}{\ref{Chapter:6}}}: A fuzzy logic based approach is proposed for fault diagnosis and condition monitoring in Industry 4.0 manufacturing processes. In this chapter, the diagnostic scheme is described in detail, combining the learning-based fault detection introduced in the previous chapter, signal-based fault detection such as envelope analysis of vibration data, and qualitative information on machine function. Experimental validation on a steel manufacturing plant using real process data, together with heuristic information, establishes the effectiveness of the proposed scheme.
\end{itemize}

\section{Author's Publication}
\begin{itemize}
    \item \cite{terlizzi2021vision} Terlizzi, M., Silano, G.,\textbf{ Russo, L}., Aatif, M., Basiri, A., Mariani, V.,  Glielmo, L. (2021, June). \textit{A Vision-Based Algorithm for a Path Following Problem}. In 2021 International Conference on Unmanned Aircraft Systems (ICUAS) (pp. 1630-1635). IEEE.
    \item \cite{terlizzi2021novel} Terlizzi, M.,\textbf{ Russo, L}., Picariello, E.,  Glielmo, L. (2021, July).\textit{ A novel algorithm for lane detection based on iterative tree search}. In 2021 IEEE International Workshop on Metrology for Automotive (MetroAutomotive) (pp. 205-209). IEEE.
    \item \cite{russo2021reinforcement} \textbf{Russo, L.}, Terlizzi, M., Tipaldi, M.,  Glielmo, L. (2021, September). \textit{A Reinforcement Learning approach for pedestrian collision avoidance and trajectory tracking in autonomous driving systems.} In 2021 5th International Conference on Control and Fault-Tolerant Systems (SysTol) (pp. 44-49). IEEE.
    \item \cite{russo2023learning}\textbf{ Russo, L.}, Nair, S. H., Glielmo, L.,  Borrelli, F. (2023). \textit{Learning for Online Mixed-Integer Model Predictive Control with Parametric Optimality Certificates.} IEEE Control Systems Letters.
    \item \cite{russo2021fault} \textbf{Russo, L.}, Sarda, K., Glielmo, L.,  Acernese, A. (2021, October). \textit{Fault detection and diagnosis in steel industry: a one class-support vector machine approach.} In 2021 IEEE International Conference on Systems, Man, and Cybernetics (SMC) (pp. 2304-2309). IEEE.
    \item \cite{mazzoleni2022fuzzy} Mazzoleni, M., Sarda, K., Acernese, A., \textbf{Russo, L.}, Manfredi, L., Glielmo, L.,  Del Vecchio, C. (2022). \textit{A fuzzy logic-based approach for fault diagnosis and condition monitoring of industry 4.0 manufacturing processes.} Engineering Applications of Artificial Intelligence, 115, 105317.
\end{itemize}


 
  
